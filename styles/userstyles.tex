\graphicspath{{images/}}

%%% Мои стили
\DeclareMathOperator{\sgn}{sgn}
\DeclareMathOperator{\inter}{int}
\DeclareMathOperator{\id}{id}
\renewcommand*\d\partial
\newcommand*{\bint}{\displaystyle\int}
\newcommand*{\oprint}[2]{\displaystyle\int\limits_{#1}^{#2}}
\newcommand*{\smalloprint}[2]{\int\limits_{#1}^{#2}}
\newcommand*\R{\mathbb{R}}
\newcommand*\Z{\mathbb{Z}}
\newcommand*\N{\mathbb{N}}
\newcommand*\M{\mathbb{M}}
\newcommand*\Q{\mathbb{Q}}
\renewcommand*\C{\mathbb{C}}
\newcommand*\grad\nabla

%%% Новые окружения для определений, теорем и т. д.
\theoremstyle{definition} \newtheorem{theorem}{\scshape Теорема}[chapter]
\theoremstyle{definition} \newtheorem{lemma}[theorem]{\scshape Лемма}
\theoremstyle{definition} \newtheorem{definition}[theorem]{\scshape Определение}
\theoremstyle{definition} \newtheorem{conjecture}[theorem]{\scshape Утверждение}
\theoremstyle{definition} \newtheorem{example}[theorem]{\scshape Пример}
\theoremstyle{definition} \newtheorem{corollary}[theorem]{\scshape Следствие}
\theoremstyle{definition} \newtheorem{algorithm}{Алгоритм}[section]
\renewenvironment{proof}{{\scshape Доказательство.}}{\qed}
\newcommand{\indention}[1]{\vspace{1.5em}\textbf{#1}}

%%% Номера задач включают номер главы
\renewcommand*{\theenumi}{\thechapter.\arabic{enumi}}

%%% Русская традиция начертания математических знаков
\renewcommand{\le}{\ensuremath{\leqslant}}
\renewcommand{\leq}{\ensuremath{\leqslant}}
\renewcommand{\ge}{\ensuremath{\geqslant}}
\renewcommand{\geq}{\ensuremath{\geqslant}}
\renewcommand{\emptyset}{\varnothing}

%%% Русская традиция начертания математических функций (на случай копирования из зарубежных источников)
\renewcommand{\tan}{\operatorname{tg}}
\renewcommand{\cot}{\operatorname{ctg}}
\renewcommand{\csc}{\operatorname{cosec}}

%%% Русская традиция начертания греческих букв (греческие буквы вертикальные, через пакет upgreek)
\renewcommand{\epsilon}{\ensuremath{\upvarepsilon}}   %  русская традиция записи
\renewcommand{\phi}{\ensuremath{\upvarphi}}
%\renewcommand{\kappa}{\ensuremath{\varkappa}}
\renewcommand{\alpha}{\upalpha}
\renewcommand{\beta}{\upbeta}
\renewcommand{\gamma}{\upgamma}
\renewcommand{\delta}{\updelta}
\renewcommand{\varepsilon}{\upvarepsilon}
\renewcommand{\zeta}{\upzeta}
\renewcommand{\eta}{\upeta}
\renewcommand{\theta}{\uptheta}
\renewcommand{\vartheta}{\upvartheta}
\renewcommand{\iota}{\upiota}
\renewcommand{\kappa}{\upkappa}
\renewcommand{\lambda}{\uplambda}
\renewcommand{\mu}{\upmu}
\renewcommand{\nu}{\upnu}
\renewcommand{\xi}{\upxi}
\renewcommand{\pi}{\uppi}
\renewcommand{\varpi}{\upvarpi}
\renewcommand{\rho}{\uprho}
%\renewcommand{\varrho}{\upvarrho}
\renewcommand{\sigma}{\upsigma}
%\renewcommand{\varsigma}{\upvarsigma}
\renewcommand{\tau}{\uptau}
\renewcommand{\upsilon}{\upupsilon}
\renewcommand{\varphi}{\upvarphi}
\renewcommand{\chi}{\upchi}
\renewcommand{\psi}{\uppsi}
\renewcommand{\omega}{\upomega}

\def\slantfrac#1#2{ \hspace{3pt}\!^{#1}\!\!\hspace{1pt}/
    \hspace{2pt}\!\!_{#2}\!\hspace{3pt}
} %Макрос для красивых дробей в строчку (например, 1/2)

%%% Заголовки %%%
\titleformat{\chapter}[display]
    {\normalfont\Large\filcenter\sffamily}
    {\titlerule[1pt]%
    \vspace{1pt}%
    \titlerule
    \vspace{1pc}%
    \Large\MakeUppercase{\chaptertitlename} \thechapter}
    {1pc}
    {\titlerule
    \vspace{1pc}%
    \Huge}

\titleformat{\section}[block]
    {\normalfont\Large\filcenter\sffamily}
    {\Large\thesection.}{0.5em}{}
    
%%% Колонтитулы %%%
\setlength{\headheight}{15pt}

\pagestyle{fancy}
\renewcommand{\chaptermark}[1]{ \markboth{\thechapter.~~\MakeUppercase{#1}}{}}
\renewcommand{\sectionmark}[1]{ \markright{\thesection.~~\MakeUppercase{#1}}}

\fancyhf{}
\fancyhead[LE,RO]{{\small\sffamily\thepage}}
\fancyhead[CE]{{\small\bf\sffamily\leftmark}}
\fancyhead[CO]{{\small\bf\sffamily\rightmark}}

\fancypagestyle{plain}{ %
  \fancyhf{} % remove everything
  \renewcommand{\headrulewidth}{0pt} % remove lines as well
  \renewcommand{\footrulewidth}{0pt}
}